\section{Syllabus: ME 7120: Finite Element Method Applications, Fall
2016}\label{syllabus-me-7120-finite-element-method-applications-fall-2016}

\subsection{Instructor}\label{instructor}

\href{http://www.cs.wright.edu/~jslater}{Joseph C. Slater, PhD, PE}\\
209 Russ Engineering Center\\
775-5005

\subsection{Text}\label{text}

Cook, R.D., Malkus, D.S., and Plesha, M.E., and Witt, R.J., Concepts and
Applications of Finite Element Analysis, 4th Edition, Wiley, 2001.
Supplementary texts are listed on the course web page.

\subsection{Office Hours}\label{office-hours}

Tentative: Will change depending on student schedules. 3:30-4:30
Tuesday/Thursday, and by appointment. You are encouraged to use email to
contact me when you have questions. You \emph{must} use your Wright
State Email to ensure that it makes it through spam filters to my in
box! You will get a quicker response by email than by any other mode of
communication.

\subsection{Software}\label{software}

ANSYS via your engineering UNIX account or another commercial FEA code
when required for homework/projects. You may obtain an account by going
to the Mechanical and Materials Engineering department office and
requesting a graduate student account on the engineering UNIX systems.

Programming will be done in Matlab. Please refer to the extensive Matlab
help menu, including tutorials etc. Simple introductions are available
on my website, and Google is also capable of answering almost any
question a new user has. Matlab is free to students through the
CATS website \textless{}https://www.wright.edu/information-technology/services/matlab-software\textgreater{}.

`WFEM
\textless{}\url{http://www.cecs.wright.edu/people/faculty/jslater/classes/fem2/WFEM/wfem.zip}\textgreater{}: You are required to learn how this code works and use it as a
framework for your projects. Read the included WFEM.pdf manual. It has many parts of necessary analysis
already completed and generalized so that you may focus solely on
generation of element matrices and assembly. An included example element can be used as a template for your code. No other files
should be modified (broken, if you are not careful!). It is extremely unlikely that you will find
an undiscovered error impeding your project. {[}1{]}\_

Prerequisites
-\/-\/-\/-\/-\/-\/-\/-\/-\/-\/-\/-\/-

ME 4120/6120. If you have not taken ME 4120 or 6120, or a senior level
finite element course at another university, you will not be able to
keep up. 

Attendance
-\/-\/-\/-\/-\/-\/-\/-\/-\/-

Attendance is optional and you are not assigned any grade for
attendance. If you miss class, you are responsible for missed material,
including assignments, from another student. I will not repeat with you
material covered in class if you do not attend.

Course Contents
-\/-\/-\/-\/-\/-\/-\/-\/-\/-\/-\/-\/-\/-\/-

+-\/-\/-\/-\/-\/-\/-\/-\/-\/-\/-\/-\/-\/-\/-\/-\/-\/-+-\/-\/-\/-\/-\/-\/-\/-\/-\/-\/-\/-\/-\/-\/-\/-\/-\/-\/-\/-\/-\/-\/-\/-\/-\/-\/-\/-\/-\/-\/-\/-\/-\/-\/-\/-\/-\/-\/-\/-\/-\/-\/-\/-\/-\/-\/-\/-\/-\/-\/-\/-\/-\/-\/-\/-\/-\/-\/-+
\textbar{} Week number(s)   \textbar{} Topic                                                     \textbar{}
+-\/-\/-\/-\/-\/-\/-\/-\/-\/-\/-\/-\/-\/-\/-\/-\/-\/-+-\/-\/-\/-\/-\/-\/-\/-\/-\/-\/-\/-\/-\/-\/-\/-\/-\/-\/-\/-\/-\/-\/-\/-\/-\/-\/-\/-\/-\/-\/-\/-\/-\/-\/-\/-\/-\/-\/-\/-\/-\/-\/-\/-\/-\/-\/-\/-\/-\/-\/-\/-\/-\/-\/-\/-\/-\/-\/-+
\textbar{} 1                \textbar{} Review of Finite Elements                                 \textbar{}
+-\/-\/-\/-\/-\/-\/-\/-\/-\/-\/-\/-\/-\/-\/-\/-\/-\/-+-\/-\/-\/-\/-\/-\/-\/-\/-\/-\/-\/-\/-\/-\/-\/-\/-\/-\/-\/-\/-\/-\/-\/-\/-\/-\/-\/-\/-\/-\/-\/-\/-\/-\/-\/-\/-\/-\/-\/-\/-\/-\/-\/-\/-\/-\/-\/-\/-\/-\/-\/-\/-\/-\/-\/-\/-\/-\/-+
\textbar{} 2                \textbar{} Elasticity Theory                                         \textbar{}
+-\/-\/-\/-\/-\/-\/-\/-\/-\/-\/-\/-\/-\/-\/-\/-\/-\/-+-\/-\/-\/-\/-\/-\/-\/-\/-\/-\/-\/-\/-\/-\/-\/-\/-\/-\/-\/-\/-\/-\/-\/-\/-\/-\/-\/-\/-\/-\/-\/-\/-\/-\/-\/-\/-\/-\/-\/-\/-\/-\/-\/-\/-\/-\/-\/-\/-\/-\/-\/-\/-\/-\/-\/-\/-\/-\/-+
\textbar{} 3-4              \textbar{} Variational Methods                                       \textbar{}
+-\/-\/-\/-\/-\/-\/-\/-\/-\/-\/-\/-\/-\/-\/-\/-\/-\/-+-\/-\/-\/-\/-\/-\/-\/-\/-\/-\/-\/-\/-\/-\/-\/-\/-\/-\/-\/-\/-\/-\/-\/-\/-\/-\/-\/-\/-\/-\/-\/-\/-\/-\/-\/-\/-\/-\/-\/-\/-\/-\/-\/-\/-\/-\/-\/-\/-\/-\/-\/-\/-\/-\/-\/-\/-\/-\/-+
\textbar{} 4-5              \textbar{} General Derivation of Linear Finite Element Formulation   \textbar{}
+-\/-\/-\/-\/-\/-\/-\/-\/-\/-\/-\/-\/-\/-\/-\/-\/-\/-+-\/-\/-\/-\/-\/-\/-\/-\/-\/-\/-\/-\/-\/-\/-\/-\/-\/-\/-\/-\/-\/-\/-\/-\/-\/-\/-\/-\/-\/-\/-\/-\/-\/-\/-\/-\/-\/-\/-\/-\/-\/-\/-\/-\/-\/-\/-\/-\/-\/-\/-\/-\/-\/-\/-\/-\/-\/-\/-+
\textbar{} 6-7              \textbar{} Isoparametric 1-D Elements                                \textbar{}
+-\/-\/-\/-\/-\/-\/-\/-\/-\/-\/-\/-\/-\/-\/-\/-\/-\/-+-\/-\/-\/-\/-\/-\/-\/-\/-\/-\/-\/-\/-\/-\/-\/-\/-\/-\/-\/-\/-\/-\/-\/-\/-\/-\/-\/-\/-\/-\/-\/-\/-\/-\/-\/-\/-\/-\/-\/-\/-\/-\/-\/-\/-\/-\/-\/-\/-\/-\/-\/-\/-\/-\/-\/-\/-\/-\/-+
\textbar{} 7-8              \textbar{} Isoparametric 3-D Elements-Bricks                         \textbar{}
+-\/-\/-\/-\/-\/-\/-\/-\/-\/-\/-\/-\/-\/-\/-\/-\/-\/-+-\/-\/-\/-\/-\/-\/-\/-\/-\/-\/-\/-\/-\/-\/-\/-\/-\/-\/-\/-\/-\/-\/-\/-\/-\/-\/-\/-\/-\/-\/-\/-\/-\/-\/-\/-\/-\/-\/-\/-\/-\/-\/-\/-\/-\/-\/-\/-\/-\/-\/-\/-\/-\/-\/-\/-\/-\/-\/-+
\textbar{} 9                \textbar{} Isoparametric 3-D Elements-Tetrahedrons                   \textbar{}
+-\/-\/-\/-\/-\/-\/-\/-\/-\/-\/-\/-\/-\/-\/-\/-\/-\/-+-\/-\/-\/-\/-\/-\/-\/-\/-\/-\/-\/-\/-\/-\/-\/-\/-\/-\/-\/-\/-\/-\/-\/-\/-\/-\/-\/-\/-\/-\/-\/-\/-\/-\/-\/-\/-\/-\/-\/-\/-\/-\/-\/-\/-\/-\/-\/-\/-\/-\/-\/-\/-\/-\/-\/-\/-\/-\/-+
\textbar{} 10               \textbar{} Plate Elements                                            \textbar{}
+-\/-\/-\/-\/-\/-\/-\/-\/-\/-\/-\/-\/-\/-\/-\/-\/-\/-+-\/-\/-\/-\/-\/-\/-\/-\/-\/-\/-\/-\/-\/-\/-\/-\/-\/-\/-\/-\/-\/-\/-\/-\/-\/-\/-\/-\/-\/-\/-\/-\/-\/-\/-\/-\/-\/-\/-\/-\/-\/-\/-\/-\/-\/-\/-\/-\/-\/-\/-\/-\/-\/-\/-\/-\/-\/-\/-+
\textbar{} 11-13            \textbar{} Eigen-solution Methods                                    \textbar{}
+-\/-\/-\/-\/-\/-\/-\/-\/-\/-\/-\/-\/-\/-\/-\/-\/-\/-+-\/-\/-\/-\/-\/-\/-\/-\/-\/-\/-\/-\/-\/-\/-\/-\/-\/-\/-\/-\/-\/-\/-\/-\/-\/-\/-\/-\/-\/-\/-\/-\/-\/-\/-\/-\/-\/-\/-\/-\/-\/-\/-\/-\/-\/-\/-\/-\/-\/-\/-\/-\/-\/-\/-\/-\/-\/-\/-+
\textbar{} 13-14            \textbar{} Dynamic Transient Response                                \textbar{}
+-\/-\/-\/-\/-\/-\/-\/-\/-\/-\/-\/-\/-\/-\/-\/-\/-\/-+-\/-\/-\/-\/-\/-\/-\/-\/-\/-\/-\/-\/-\/-\/-\/-\/-\/-\/-\/-\/-\/-\/-\/-\/-\/-\/-\/-\/-\/-\/-\/-\/-\/-\/-\/-\/-\/-\/-\/-\/-\/-\/-\/-\/-\/-\/-\/-\/-\/-\/-\/-\/-\/-\/-\/-\/-\/-\/-+



Homework (10\%)
-\/-\/-\/-\/-\/-\/-\/-\/-\/-\/-\/-\/-\/-

Homework problems will be due every Friday in class and is checked
only for effort (that you did it) and on your presentation of your
solution to class (as tasked). Late homework is not be accepted.
Homework is updated regularly on the course webpage. It represents a
minimal amount of problems necessary to understand the material.
Students are expected to work additional problems for self-study.

Labs (40\%)
-\/-\/-\/-\/-\/-\/-\/-\/-\/-

Labs (projects) must be completed successfully to complete the course.
You are expected to start on them early so that guidance given in class
is better understood at the time it is given. Insights given in class
will be helpful in debugging only if you have progressed to the point in
the project that you can understand the points made.

Exams (50\%)
-\/-\/-\/-\/-\/-\/-\/-\/-\/-\/-

There will be one test and a final exam graded on a straight, scale
(:math:geq 90 = A,geq 80 = B, geq 70 = C, geq 60 = D, \textless{} 60 =
F`). The final exam will count for two test grades. The lowest exam
grade of the three will be dropped. If you know during the first week of
the semester that you cannot attend an exam, please see me. All grading
discrepancies must be brought up in writing no later than one week after
the exam is returned. A simple note describing your contentions will do.

\subsection{Problem Solutions}\label{problem-solutions}

All problem solutions, whether on homework, labs, or exams, should be
neat and orderly.

\subsection{Important Dates}\label{important-dates}

\begin{longtable}[]{@{}ll@{}}
\toprule
\begin{minipage}[t]{0.34\columnwidth}\raggedright\strut
October 3:\strut
\end{minipage} & \begin{minipage}[t]{0.35\columnwidth}\raggedright\strut
Midterm 1\strut
\end{minipage}\tabularnewline
\begin{minipage}[t]{0.34\columnwidth}\raggedright\strut
November 7:\strut
\end{minipage} & \begin{minipage}[t]{0.35\columnwidth}\raggedright\strut
Midterm 2\strut
\end{minipage}\tabularnewline
\begin{minipage}[t]{0.34\columnwidth}\raggedright\strut
December 16:\strut
\end{minipage} & \begin{minipage}[t]{0.35\columnwidth}\raggedright\strut
Final Exam\strut
\end{minipage}\tabularnewline
\bottomrule
\end{longtable}
