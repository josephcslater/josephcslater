\newcommand*\Q[2]{\frac{\partial #1}{\partial #2}}

\section<presentation>*{Übersicht}
\begin{frame}{Übersicht}  \tableofcontents[part=1,pausesections] \end{frame}

\AtBeginSubsection[]{\begin{frame}<beamer>
    \frametitle{Übersicht} \tableofcontents[current,currentsubsection] \end{frame} }

\part<presentation>{Hauptteil}

\section{Forschung und Studium}
\begin{frame}{Das Integral und seine geometrischen Anwendungen.} 
Die erste Gleichung von Green:
\begin{align}\label{green}
\underset{\mathcal{G}\quad}\iiint\!
	\left[u\nabla^{2}v+\left(\nabla u,\nabla v\right)\right]d^{3}V
	=\underset{\mathcal{S}\quad}\oiint u\Q{v}{n}d^{2}A
\end{align}

Die Gleichung von Green (\ref{green}) wird später überpüft.

\begin{itemize}
  \item Eine Zeile mit \texttt{itemize}.
  \begin{itemize}
    \item Eine Zeile mit \texttt{itemize}.
    \begin{enumerate}
      \item Eine Zeile mit \texttt{enumerate}.
      \item Noch eine \ldots
    \end{enumerate}
    \item Eine Zeile mit \texttt{itemize}.
  \end{itemize}
  \item Eine Zeile mit \texttt{itemize}.
\end{itemize}
\end{frame}
\subsection{Intervall}
\begin{frame}{Definition}
Das \emph{Intervall} $\langle a,b\rangle$ besteht aus allen Zahlen $x$, die den 
Bedingungen $<\le x\le b$  genügen.
\end{frame}
\subsection{Zahlenfolge}
\begin{frame}{Definition der Folge}
Eine \emph{Zahlenfolge} oder \emph{Folge} entsteht, wenn man sich jedes Glied der 
unendlichen Nummernreihe $1,2,3,\ldots$ durch irgend eine (rationale oder irrationale) 
Zahl ersetzt denkt, also jedes $n$ durch eine Zahl $x_n$.
\end{frame}
\subsection{Limes}
\begin{frame}{Definition Limes}
$\lim x_n=g$ bedeutet, daß in jeder Umgebung von $g$ fast alle Glieder der Folge liegen.
\end{frame}
\subsection{Konvergenzkriterium}
\begin{frame}{Definition der Konvergenz}
\textbf{Konvergenzkriterium}. Die Folge $x_1,x_2,x_3,\ldots$ ist dann und nur dann 
konvergent, wenn \textbf{jede} Teilfolge $x^\prime_1,x^\prime_2, x^\prime_3,\ldots$ die 
Relation $\lim(x_n-x^\prime_n)=0$
\end{frame}

\endinput