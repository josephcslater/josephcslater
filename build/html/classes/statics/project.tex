\documentclass[11pt]{article}

\begin{document}

% Definition of title page:
\title{
    Design Project for ME 212 (Statics)
}
\author{
   Preliminary Report (2 pages, at least one design iteration.)\\ 
   Due Oct.~26- ME Department Mailbox\\
Final Due date: Nov.~9 (No extensions)- ME Department Mailbox\\
Groups of four or less     % insert author(s) here
}

\maketitle

The task of designing a one-lane bridge for immediate construction has been assigned to your group.  You are competing with other companies for the contract, and as a result, want to generate as competitive a proposal as possible.  Only two types of beams and one type of cable are available for construction of the bridge.  The reason for this restriction is to simplify construction and minimize the complexity that occurs when a large number of parts must be tracked and supplied. 

Your task is to design the side of the bridge (a 2-D truss) and a support for the road bed using the available materials such that the load carrying capacity of each of member is not exceeded and a sufficient additional degree of safety is provided.  The truss must extend over a 20 meter ravine and be capable of supporting a $60$ kN, $3.5$ m wide, vehicle at any location on the bridge.

The thinner beam (element type 1) has a load carrying capacity of $14.1$ kN and can be obtained in $6$ meter lengths.  The maximum internal bending moment is 40 kN-m.

The larger beam (element type 2) has a load carrying capacity of $30.4$ kN, can be obtained in $5$ meter lengths, and has a cost of materials and transportation of roughly twice that of the smaller beams.  The maximum internal bending moment is $160$ kN-m.

The cable has a load carrying capacity of $27.4$ kN, can be obtained in $60$ meter lengths, and is comparable in cost to the smaller beams for a $7$ m section.

You may assume that $1.5$ m $ \times 1.5$ m square rigid plate are available and capable of supporting 20 kN anyplace on their surface.  The cost of the plates is roughly $1/4$ that of the thinner beam.

Consider safety, ease of assembly, cost, and possible variation in material quality in your design.  Take three full \emph{typewritten} pages (double spaced) to discuss the strengths and weaknesses of your design with respect to the aforementioned considerations.  This is in addition to mechanical sketches and analysis.  The maximum load in each member of the truss must be displayed in a clearly labeled figure.  All of the design sketches you make and their analyses should be in the appendix.  In addition, the report should clearly indicate what parts of the project were performed by each member of the group.  Additionally, each member of the group is required to turn in a confidential evaluation of each of the other members of the group to me at the completion of the project.

Which member in your design do you expect to fail first if the bridge is overloaded?  What is your safety factor?

The safety factor of the bridge is the smallest ratio of the load capability of a member divided by the greatest load it will sustain. Anything below zero is a non-working bridge.

Be sure to analyze the bridge for appropriate loading at any location on the bridge so that the determined maximum expected loads in each member are appropriate.

\section*{Proposal Format}

The report must contain the following items:
\begin{itemize}
\item Title page: Design title, proposal authors
\item Overview 
\item Design details: bridge specifications, size, cost, safety 
factors of members, drawings of bridge, strengths and weaknesses of 
your design
\item Present the loads, load capabilities, and safety factor for 
each member in table form.
\item Conclusion: Brief summary of design. 
\item Appendix: Details of analysis describing method of analysis. At 
least three working design iterations, showing analysis of each 
truss. They do not have to be different types of bridge, they must 
simply have different node locations.
\end{itemize}


\section*{Pre-proposal}
At least one working bridge design, with solution illustrating that it works.


\end{document}
% Local Variables:
% TeX-header-end: "% End-Of-Header"
% TeX-trailer-start: "% Start-Of-Trailer"
% TeX-command-default: "pdfLaTeX"
% TeX-master: t
% End:
