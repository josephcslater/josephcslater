\documentclass[11pt]{article}

\usepackage{amsmath}
\setlength{\topmargin}{0in}
\setlength{\headheight}{0in}
\setlength{\headsep}{0in}
\setlength{\textheight}{9in}
\begin{document}
\begin{center}
	\LARGE{ME 712 Lab 1: 3-Dimensional Brick Element}
	Due May 30
\end{center}
\vspace{.3in}
\begin{enumerate}

\item Write a general code to generate the elemental stiffness matrix for a single  brick element for WFEM.  

\item Write a code to generate the elemental mass matrix.

\item Obtain the FE matrices $K$ in global coordinates.

\item Write a subroutine to assemble these elements into a global matrix.

\item Write a subroutine to apply specified boundary conditions (imposed 
displacements).

\item Write a subroutine to solve for the entire displacement vector.

\item Test your code on the following:\label{it2} 
Find the tip displacement of a vertical post to a 10nN transverse tip load for the following structures:
\begin{enumerate}
	\item Pyramid: height 190nm, bottom side length 74nm, flat top side length 10nm
	\item Height 190nm, bottom diameter 74nm, flat top diameter 10nm.
\end{enumerate}


\item Compare  your results to a beam model using your beam code, and results using your favorite commercial finite element code. Explain discrepancies. 

\item Work in groups of two. 
	
	

\end{enumerate}

%\\

%\noindent Bonus: Write a subroutine to take elemental displacements and return the 
%direct stress, maximum bending moment in the beam, maximum shear 
%load, and torque in the beam.

\noindent Be sure to write your code in a modular form to make debugging easier. 
Make sure to comment your code as well.  Turn in the results of all of you
analyses as well as the code. tar.zip (tgz), or PC zip all files and email them to me.

\end{document}
