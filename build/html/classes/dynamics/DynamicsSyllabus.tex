\documentclass[12pt]{article}
%\usepackage{latexsym}
\usepackage{latexsym,times,url}
\usepackage{booktabs}
%\usepackage{enumerate}
\setlength{\oddsidemargin}{0in}
\setlength{\topmargin}{-.5in}
\setlength{\baselineskip}{18pt}
\setlength{\textwidth}{6.5in}
\setlength{\textheight}{9in}
\usepackage[colorlinks=true]{hyperref}
%\hypersetup{colorlinks=true}
\newcommand{\comp}[1]{{\small\textsf{#1}}}
\title{ME 213: Dynamics}
\author{}
\date{Winter 2010}

\usepackage{sectsty,fancyhdr,lastpage} 
\pagestyle{fancy}
\cfoot{\thepage\ of \pageref{LastPage}}
\chead{}
\lhead{}
\rhead{}
\renewcommand{\headrulewidth}{0pt}

\newcommand{\matlab}{\href{http://www.mathworks.com}{M{\small ATLAB}}}
\begin{document}
\maketitle


\section*{Instructor}
Joseph C.~Slater, PhD, PE\\
238 Russ Engineering Center\\
\href{http://www.cs.wright.edu/~jslater}{\url{http://www.cs.wright.edu/~jslater}}\\
775-5040

\section*{Communication}
Please see my extensive web pages for contact information. It is the most up to date. 

 Email will be sent regularly to your university email accounts.  This will allow \href{http://www.cs.wright.edu/~jslater}{me} to contact you regarding the class between lectures.  If you would rather have your email forwarded somewhere else, please \href{http://www.wright.edu/cats/info/email/emailoptions.html}{read how to do that}.  This information and the answers to most any question you may have regarding the course and computer usage (including old exams, and the syllabus) can be found on my \href{http://www.cs.wright.edu/~jslater/classes/computerhelp.shtml}{computer help page} or the course web page.
 
Because weather in Winter Quarter can be unpredictable, it is advisable that you set a filter on your university account so that any email I send to you with the text ``cancel'' in it will be \href{http://www.tech-recipes.com/rx/939/sms_email_cingular_nextel_sprint_tmobile_verizon_virgin/}{forwarded to your cell phone as an SMS text message}. 

\section*{Time and Place}
MWF: 8:30-9:35 AM, 148 Russ Hall

\section*{Office Hours}
Will change depending on student schedules. 1:00-2:00 PM, Monday and Wednesday, and by appointment. %Currently 2:45-3:45 PM Monday and Wednesday, and by appointment. Please use email to contact me when you have questions. 
I check my email many times each day. You will get a quicker response 
by {email} than by any other mode of communication.

\section*{Text}
Gray, Costanzo, Plesha, Engineering Mechanics: Dynamics, McGraw Hill, 2009.
 - Required\\
Also see the \href{http://www.cs.wright.edu/~jslater/classes/dynamics}{course web page}. 

\section*{Software}
You are expected to have a graphics calculator with linear algebra and numerical solver capabilities. 

Mathematica, \textsc{Matlab}, \href{http://www.octave.org}{Octave}, any version (student or professional version). 
These are available through your WSU account (except Octave, which you may download and use for free on your home computer), but you may find it convenient to buy at the bookstore and install on your home computer. They will run on any platform.

Please read the text and use the on-line help.  Syntax issues are sufficiently discussed or displayed in these resources.  It is highly recommended that you learn to use \href{http://www.wolfram.com}{Mathematica} and \matlab\  on your UNIX account as soon as possible. Each has strengths and weaknesses, and they can be of great help with more complex problems, especially on your project. Examples are given for all three in my \href{http://www.cs.wright.edu/~jslater/classes/materials/CompLit.pdf}{Computer Literacy course notes}. 
%\noindent Use either \comp{gandalf}, \comp{odin}, \comp{reality} or \comp{gamma},  depending on system loads (log in and use the \comp{uptime} command.  Please see the computer consultants on the 2nd floor across from the elevators if you have difficulties with your accounts.

\section*{Prerequisites}
ME 212 Minimum Grade of C and (ME 102 or CEG 220)

\section*{Prerequisites by topic (5\%)}
You are expected to know the following.  You will be tested on them
the first day of the second week.  Please review your old course notes and texts:
\begin{enumerate}
\setlength{\itemsep}{0pt}
	\item Definite Integrals
	\item Polar Coordinates
	\item Parametric Equations
	\item Vector Algebra
	\item Force Systems
	\item Equilibrium
	\item Structures
	\item Distributed Forces
	\item Friction
	\item Mass Moment of Inertia
	\item Parallel Axis Theorem
	\item Mechanics of Particles and Rigid Bodies (PHY 240)
\end{enumerate}

\section*{Course Contents}

At the successful conclusion of this course, students will be able to:
\begin{itemize}
\item solve problems involving kinematics of particles.
\item solve problems using kinetics of particles, Newton�s laws, momentum, and energy methods.
\item solve problems applying kinetics of systems of particles.
\item solve problems by applying kinematics of rigid bodies (2-D).
\item solve problems by applying Newton�s laws for 2-D bodies.
\item solve problems by applying energy and momentum methods for 2-D rigid bodies.
\end{itemize}
\subsection*{Lecture Schedule}
\begin{center}
   %\topcaption{Table captions are better up top} % requires the topcapt package
   \begin{tabular}{llll} % Column formatting, @{} suppresses leading/trailing space
      %\toprule
     &&Day&\\
      \cmidrule(r){2-4} % Partial rule. (r) trims the line a little bit on the right; (l) & (lr) also possible
      Week&Monday&Wednesday& Friday\\
      1 (Jan 3)&1.1-2.1&2.2-2.3&2.4-2.5\\
      2 (Jan 10)&2.6-2.7&2.8-3.1&3.2\\
      3 (Jan 17)&X&3.3&E\\
      4 (Jan 24)&4.1-4.2&4.3&4.4\\
      5 (Jan 31)&5.1&5.2&5.3\\
      6 (Feb 7)&5.5&E&6.1\\
      7 (Feb 14)&6.2&6.3&S\\
      8 (Feb 21)&6.4&Appendix A&7.1\\
      9 (Feb 28)&7.1 (cont)&E&8.1\\
      10 (Mar 7)&8.2&8.3&Review\\
\end{tabular}
 \end{center}

% Requires the booktabs if the memoir class is not being used
\subsection*{Major Dates}
\begin{center}
   %\topcaption{Table captions are better up top} % requires the topcapt package
   \begin{tabular}{ll} % Column formatting, @{} suppresses leading/trailing space
Date&Event\\
\hline
Jan 11& Prerequisite Quiz\\
Jan 18& Martin Luther King day (no classes)\\
Jan 22&Exam I\\
Feb 10& Exam II\\
Feb 19& Service day\\
Feb 26& Pre-proposal (project draft) due\\
March 3&Exam III\\
March 12& Project due\\
March 19 (8:30-10:30)& Final Exam
  \end{tabular}
\end{center}



\subsection*{Homework (10\%)}
Each homework problem is worth 1 point.  You are encouraged to work together in small groups, but keep in mind that homework is assigned in order to help you learn and keep up with the course material.  You are expected to learn how to use your \emph{matrix capable, numerical solution capable, graphics calculator}\footnote{This is a tool invented in the 20th century. Please obtain one before we are too far into the 21st.} and \matlab\ in order to solve more complex problems.  This will benefit you during quizzes.  You are also encouraged to do additional problems out of the text for practice.  The assigned problems are the \emph{minimum} necessary to master the material.  \emph{The only way to learn the skills taught in this course is to apply them.} Homework may not be turned in late \footnote{Solutions covered in class immediately after homework is due.  It's not fair to others to hold up going over solutions because of a few late assignments.  It is strongly encouraged that you complete late assignments for your own benefit.}  Please see me if you need help with the homework.  Homework grades, unlike other grades,  will be curved such that the average homework grade is a \emph{B for problems turned in}.
\subsection*{Homework (posted by due date)}
This homework is a minimum amount of homework to understand the material. Only exhaustive solution of the problems in the text can guarantee mastery of the material. 
\begin{center}
   %\topcaption{Table captions are better up top} % requires the topcapt package
   \begin{tabular}{lp{2in}ll} % Column formatting, @{} suppresses leading/trailing space
      %\toprule
     &&Day&\\
      \cmidrule(r){2-4} % Partial rule. (r) trims the line a little bit on the right; (l) & (lr) also possible
      Week&Monday&Wednesday& Friday\\
      %\hline\\
      1 (Jan 3)&-&1.3, 1.5, 1.8, 2.2, 2.10&2.15, 2.24, 2.50 \\
      2 (Jan 10)&\href{http://www.cs.wright.edu/~jslater/classes/dynamics/Extra}{Statics Final}, 2.79, 89, 2.120, 2.138, 2.143&2.161, 2.172, 2.193&2.222, 2.249, 3.4\\
      3 (Jan 17)&X&3.5,3.6, 3.22,3.47&E\\
      4 (Jan 24)&3.91, 3.94, 3.109&4.7, 4.14, 4.17&4.30, 4.36, 4.39\\
      5 (Jan 31)&4.54, 4.59, 4.72, 4.78&5.5, 5.12, 5.23&5.50, 5.58\\
      6 (Feb 7)&5.66, 5.77, 5.87 5.90&E&5.113, 5.119, 5.128\\
      7 (Feb 14)&6.17, 6.20, 6.27&6.53, 6.55&S\\
      8 (Feb 21)&6.78, 6.84, 6.87, 6.98&6.112, 6.120&Statics Chap 10: 82, 85, 91\\
      9 (Feb 28)&7.10, 7.16, 7.18&E&7.22, 7.25\\
      10 (Mar 7)&7.35, 8.14, 8.25, 8.43&8.66, 8.69&8.82, 8.88\\
      %\bottomrule
   \end{tabular}
   %\caption{Remember, \emph{never} use vertical lines in tables.}
   %\label{tab:booktabs}
\end{center}
X: No Class\\
E: Exam\\
S: Service day

\subsection*{Professionalism (5\%)}
Professionalism is a measure of your behavior regarding expected practice as an engineer. This includes aspects such as attendance, note taking, consistency of performance, tenacity in problem solution, leadership, legibility and organization of problem solutions, clarity of communication, etc. For details on expected behavior, please consult \emph{The Unwritten Rules of Engineering} by W.J.~King, with revision by  J.G.~Skakoon. This book is available at the library. However, for your own professional development, I highly recommend that you \href{http://members.asme.org/catalog/ItemView.cfm?ItemNumber=801624}{own a personal copy}.\footnote{If you read an older edition of the book (prior to Skakoon), please be attentive to the fact that some of the comments, especially those regarding polishing shoes, are considered rather quaint today. Appearance is not quite as important today is it was then.} One point will be deducted from your professionalism score each time you receive less than a 70 on an exam \emph{and} do not see me personally during the next week to clear up confusion. 

\subsection*{Design Project (10\%)}
A design project will be assigned during the quarter. You may not consult other members of the class (outside of your group) for assistance on the project. You may (and are expected to) consult the library and the instructor.

The design project is set up as a competition. You will have to write a pre-proposal, which forms, essentially, a draft of your design report. I will provide you feedback on report writing within one week. Your final proposal will be ranked against performance of this and previous classes. Only one project may `win' a score of 100. Other projects will be relegated to scores of 95 and below. Any information you generate or obtain should b treated as proprietary because this \emph{is} a competition.  

At the conclusion of the project each member of the group is advised to grade the other members of the group. You have 100 points to divide amongst your fellow group members and yourself. I use these ratings to compensate for individuals who do not pull their own weight in a group, or excel within a group. 

\subsection*{Exams (70\%)}
There are 3 midterms and a final exam. The final counts as 3 exams. You may drop the lowest 2. Your final grade will be no less than one letter grade below your final exam grade, presuming you pass the project. If you receive a passing grade on the  project and a 100 on the final exam, you will receive an `A' in the class. 

\subsection*{Problem Solutions}
Problem solutions must be neat and orderly.  They must include each of the following, when applicable.  You must be capable of making that decision.  For example: To use Newton's law, you must draw a free-body diagram.  Free body diagrams don't make sense when not applying either Newton's laws or variants of Newton's laws, i.e.~sketches for determining properties of areas, etc. All work must be done on 8.5 in.~by 11 in.~paper.
\begin{enumerate}
        \item What are you looking for?  Briefly.  For example: \emph{The force in
        cable $AB$}.

        \item \label{it:laws} What laws apply?  What principles apply?  For
        example: Newton's law-translation (alternatively can be stated in
        equation form: $\sum{\mathbf{F}}=\mathbf{0}$).  You must list all laws
        and principles that you will apply in solving the problem. 
        Equilibrium will be the most common in Statics.

        \item Sketch \emph{and/or} free body diagram (FBD).  You must have at
        least one except under the most unusual circumstances where neither applies.

        \item List of known quantities, list of \emph{all} vectors to be used,
        including the knowns and unknown variable in the vectors.  This is an
        organizational toolkit for substituting into item \ref{it:laws}.

        \item Generation of the set of equations with substitutions made. 
        Solution if simple, set of equations to be solved if not.  Solution is
        always expected on homework, using Mathematica at your discretion. 
        Expectation of solution on quizzes will be declared.  If solution is
        not expected on a quiz, the set of equations that must be solved, and
        the procedure for solving them must be stated.

        \item When you generate a solution, \emph{does the result make sense}? 
        As engineers, you must have insight as to whether or not an answer
        seems reasonable \emph{before} you solve the problem.  How long is a
        table?  Two to 12 feet would be a reasonable range.  If you get three
        miles long, you should know that the answer is wrong.
\end{enumerate}

A \href{http://www.cs.wright.edu/~jslater/classes/materials/frmt.pdf}{template} is available for use at the start of the quarter, while problems are short enough to fit the template. \textbf{You are required to use this template until instructed otherwise.}

\subsection*{Cheating}
\emph{\textbf{Don't}}. I helped write \href{http://www.wright.edu/students/judicial/integrity.html}{the university rules}, and I will pursue them when warranted. 

\emph{Cheating is defined as}: Copying the solution of a problem from any source. Using any source other than specified during a quiz or exam to solve the problem. Quizzes will be out of the book, but \emph{other than the page of the problem, the book may not be used to solve the problem}. Taking credit for work that was copied from another source is plagiarism and considered cheating. 


%To be decided.
%\begin{tabular}{ll}
%\end{tabular}





\end{document}